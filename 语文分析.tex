\documentclass{ctexart}
\usepackage{geometry}
\usepackage{fontspec}
\usepackage{tikz}
\usepackage{hyperref}  
\usepackage{CJKfntef}

\hypersetup{hidelinks,
	colorlinks=true,
	allcolors=black,
	pdfstartview=Fit,
	breaklinks=true
}

\newcommand{\mybox}[1]{\tikz[baseline=(MeNode.base)]{\node[rounded corners, fill=gray!20](MeNode){#1};}}
\geometry{papersize={21cm,29.7cm}}
\geometry{left=2.8cm,right=2cm,top=2cm,bottom=3cm}
\pagestyle{headings}
\setCJKmonofont{LXGW WenKai Mono}
\ctexset {
   abstractname = {本文概要},
   today = big,
   section/name = {第,节},
}

\newcommand{\df}{\large \mybox}
\newcommand{\nm}{\normalsize}
\newcommand{\blk}{\vspace*{1\baselineskip} }
\newcommand{\blkz}{\vspace*{2\baselineskip} }
\newcommand{\blkx}{\vspace*{4\baselineskip} }
\newcommand{\blkc}{\vspace*{6\baselineskip} }
\newcommand{\blkd}{\vspace*{10\baselineskip} }
\newcommand{\blkv}{\vspace*{16\baselineskip} }
\newcommand{\blkb}{\vspace*{32\baselineskip} }
\renewcommand{\\}{\par}

\setcounter{secnumdepth}{4}
\setcounter{tocdepth}{2}

\begin{document}


\title{语文分析}
\author{F1}
\maketitle
\linespread{1.48}

\tableofcontents
\newpage
%『』
\part{答题模板}

\section{叙述,描写与抒情}

\subsection{『段语』``这段话在文中的作用是?''}
对于提问一段话在文中的作用的题目,要注意以下几个点:\par
\df{承接上文内容概括}, \\
\df{启接下文内容概括}, \\
\df{这段话的概括}, \\
\df{这段话的文学性}\nm (如果有的话),\\
 \df{对全文情感主旨的作用}\nm 比如:凸显.

\subsection{『风格』本诗/词的风格是...}
诗词风格多种多样, 但是如果按照抒情方式分类主要有\par
\df{委婉}\nm ---寓情于景和\par
\df{豪放}\nm ---直抒胸臆\large \par
两类.

\subsection{『人形』请评点此处人物形象的描写}
对于人物形象的问题,需要注意\\
\df{神态变化}\\
\df{情感变化}\\
\df{体现了什么}\\
\subsubsection{例题}
\large 
从人物描写的角度,为甲段划线句写一段评点文字\\
\texttt{划线句使用了肖像描写和语言描写,连用动词"一怔","睁开","咧开嘴"等,生动地描绘了女人见到水生后由惊愕到欣喜再到数年以来委屈的悲涌上心头的心理变化,体现了女人对水生的思念之情感之深,反映出过往战争的日子的艰难.}\\
\nm \fangsong \\\begin{center}嘱咐\end{center}\\\begin{center}孙犁\end{center}\\太阳平西的时候,水生望着树林的疏密,辨别自家的村庄,他的家就在白洋淀边上。家近了,就要进家了!他想着许多事,父亲是不是还活着?父亲很早就有痰喘病;还有自己的女人,一别八年,分别时她肚子里正有了孩子,是不是都活着?房子被烧了吗?\\他在院子门口遇见了自己的女人。她正悄悄地关闭那外面的梢门。水生叫了一声:“你!”\\女人一怔,睁开大眼睛,咧开嘴笑了笑,就转过身子抽抽打打地哭了〔甲〕。水生看见她脚上那白布封鞋,就知道父亲准是不在了。两个人愣在那里站了一会。还是水生把门掩好,说:“不要哭了,家去吧!”他在前面走,女人在后面跟, 走到院里,女人紧走两步赶在前面,到屋里去点灯。\\他走进屋里,女人从炕上拖起一个孩子来,含泪笑着说:“来!这就是你爹, 一天价看见人家有爹,自己没爹,这不回来了。”说着已经不成声音。\\水生说:“来!我抱抱。”那孩子从睡梦里醒来,好奇地看着这个生人。 水生在黑影里问:“你叫什么?”“小平。”“几岁了?”女人转身插好门,对孩子说:“别告诉他,他不记的吗?”\\水生看着女人。离别了八年,她并没有老多少,头发虽然乱,脸孔苍白了一些,可那两只眼睛里的光,还是那么强烈。\\
女人歪在炕上,笑着问:“说真的,这八九年,你想起过我吗?” “想过。”“怎么想法?”她逼着问。\\“临过平汉路的那天夜里,我宿在一家小店,小店里有个鱼贩子是咱们乡亲。我买了一包小鱼下饭,吃着那鱼,就想起了你。”\\“胡说。还有吗?”“没有了。你知道我是出门打仗去了,不是专门想你去了。”\\“我们可常常想你,黑夜白日。”她突然支着身子坐起来,问:“你能在家住几天?”\\“就这一晚上。我是请假绕道来看你的。”“为什么不早些说?”“还没顾着啊!”\\女人呆了。她低下头去,无力地仄在炕上。过了半天,她说:“那么就赶快休息吧,明天我撑着冰床子去送你。”〔乙〕
\\鸡叫三遍,女人就起来给水生做了饭吃。这是一个大雾天,地上堆满了霜雪。女人把孩子叫醒,穿得暖暖的,背上冰床,锁了梢门,送丈夫上路。出了村,她要丈夫到爹的坟上去看看。水生说以后回来再去,女人坚持要去。她说:\\“爹叫你出去打仗了,是他一个老人家照顾了全家。这是什么日子呀?整天价东逃西窜。你不在家,爹对我们娘俩的照顾,只怕一差二错,对不起在外抗日的儿子。夜里一有风声,他就把我们叫醒。他老人家背着孩子逃跑,累的痰喘咳嗽。这些个担惊受怕的日子,把他老人家累死。还有那年大饥荒……”
\\在河边,他们上了冰床。水生坐上去,抱着孩子,用大衣给她包好脚。女人站在床子后尾,撑起了竿。女人是撑冰床的好手,她逗着孩子说:“看你爹没出息,当了八年八路军,还得叫我撑冰床子送他!”\\她轻轻地跳上冰床子后尾,像一只雨后的蜻蜓爬上草叶。轻轻用竿子向后一点,冰床子前进了。大雾笼罩着水淀,只有眼前几丈远的冰道可以望见。河两岸残留的芦苇上的霜花飒飒飘落,衣服上立时变成银白色。她用一块长的黑布紧紧把头发包住,脸冻得通红,嘴里却冒着热气。她连撑几竿,然后直起身子来,向水生一笑。小小的冰床像离开了强弩的箭,摧起的冰屑,在它前面打起团团的旋花。前面有一条窄窄的水沟,水在冰缝里汹汹地流,她只说了一声“小心”,两脚轻轻地一用劲,冰床就像受了惊的小蛇一样,抬起头来,窜过去了。\\水生提醒她说:“你慢一些,疯了?”女人擦一擦脸上的冰雪和汗,笑着说: “同志!我送你到战场上去呀,你倒说慢一些!”\\“擦破了鼻子就不闹了。”“不会。这是从小玩熟了的东西,今天更不会。在这八年里面,你知道我用这床子,送过多少次八路军?”\\冰床在霜雾里飞行。“你把我送到丁家坞,”水生说,“到那里,我就可以找到队伍了。”\\女人呆望着丈夫。停了一会,才说:“你知道,我现在心里很乱。八年才见到你,你只在家呆了不到多半夜的工夫。我为什么撑的这么块?为什么着急把你送到战场上去?我是想,快快打走敌人,你才能快快地回家。”\\冰床滑进水淀中央,这里是没有边际的冰场。太阳从冰面上升起来,冲开了雾,形成一条红色的胡同,扑到这里来,照在冰床上。女人说:“爹活着的时候常说,日本人在这里,水生出去是打开一条活路,打开了这条路,我们就能活。你记着爹的话,不要为家里的事分心,好好打仗,我等你回来。”\\在杨柳树环绕的丁家坞村边,水生下了冰床。\\女人忍住泪,笑着说:“快去吧你!记着,好好打仗,快回来,我们等着你的胜利消息。”\\ \rightline{一九四六年河间}
\songti





\subsection{『视角』之感官``请鉴赏划线句的表现力''}
此类题目灵活多变,需要灵活处理.主要需要注意\\
\df{内容}\nm 修辞,分析,情感\\
\df{结构}



\subsection{『视角』之人称``本文人称的表达效果是?''}
\df{第一人称}\nm 增强带入感,引发直观真切的体验. \\
\df{第一人称儿童视角}\nm 天真细致,引发成人后的反思.\\
\df{第二人称}\nm 跳出个人视角, 隐含了与读者的对话,拉近与读者的距离,产生与读者的共情.\\
\df{第三人称}\nm 增强叙述说理的客观性.

\subsection{『视角』之时间``随时间推移切换场景,赏析其妙''}
我们需要注意以下三点:\\
\df{环境}\\
\df{人物}\\
\df{场景}\\
\subsubsection{例题}
从``鸡叫三遍''到结束,小说随着时间推移切换场景,赏析其构思之妙.
\texttt{开头用``寒冷黎明''交代水生父亲之死 ,体现生活艰难;女人快速划冰床送水生上战场,体现了她的识时务与支持抗战;女人对水生的殷勤寄语更是反映了希望战争快速胜利的希望;最后"太阳升起""形成了红色的胡同"象征着通向胜利之路.}
\nm \fangsong
\\\centerline{原文见上}
\\鸡叫三遍,女人就起来给水生做了饭吃。这是一个大雾天,地上堆满了霜雪。女人把孩子叫醒,穿得暖暖的,背上冰床,锁了梢门,送丈夫上路。出了村,她要丈夫到爹的坟上去看看。水生说以后回来再去,女人坚持要去。她说:\\“爹叫你出去打仗了,是他一个老人家照顾了全家。这是什么日子呀?整天价东逃西窜。你不在家,爹对我们娘俩的照顾,只怕一差二错,对不起在外抗日的儿子。夜里一有风声,他就把我们叫醒。他老人家背着孩子逃跑,累的痰喘咳嗽。这些个担惊受怕的日子,把他老人家累死。还有那年大饥荒……”
\\在河边,他们上了冰床。水生坐上去,抱着孩子,用大衣给她包好脚。女人站在床子后尾,撑起了竿。女人是撑冰床的好手,她逗着孩子说:“看你爹没出息,当了八年八路军,还得叫我撑冰床子送他!”\\她轻轻地跳上冰床子后尾,像一只雨后的蜻蜓爬上草叶。轻轻用竿子向后一点,冰床子前进了。大雾笼罩着水淀,只有眼前几丈远的冰道可以望见。河两岸残留的芦苇上的霜花飒飒飘落,衣服上立时变成银白色。她用一块长的黑布紧紧把头发包住,脸冻得通红,嘴里却冒着热气。她连撑几竿,然后直起身子来,向水生一笑。小小的冰床像离开了强弩的箭,摧起的冰屑,在它前面打起团团的旋花。前面有一条窄窄的水沟,水在冰缝里汹汹地流,她只说了一声“小心”,两脚轻轻地一用劲,冰床就像受了惊的小蛇一样,抬起头来,窜过去了。\\水生提醒她说:“你慢一些,疯了?”女人擦一擦脸上的冰雪和汗,笑着说: “同志!我送你到战场上去呀,你倒说慢一些!”\\“擦破了鼻子就不闹了。”“不会。这是从小玩熟了的东西,今天更不会。在这八年里面,你知道我用这床子,送过多少次八路军?”\\冰床在霜雾里飞行。“你把我送到丁家坞,”水生说,“到那里,我就可以找到队伍了。”\\女人呆望着丈夫。停了一会,才说:“你知道,我现在心里很乱。八年才见到你,你只在家呆了不到多半夜的工夫。我为什么撑的这么块?为什么着急把你送到战场上去?我是想,快快打走敌人,你才能快快地回家。”\\冰床滑进水淀中央,这里是没有边际的冰场。太阳从冰面上升起来,冲开了雾,形成一条红色的胡同,扑到这里来,照在冰床上。女人说:“爹活着的时候常说,日本人在这里,水生出去是打开一条活路,打开了这条路,我们就能活。你记着爹的话,不要为家里的事分心,好好打仗,我等你回来。”\\在杨柳树环绕的丁家坞村边,水生下了冰床。\\女人忍住泪,笑着说:“快去吧你!记着,好好打仗,快回来,我们等着你的胜利消息。”
\songti


\subsection{『形象』``赏析该描写在刻画形象上的妙处''}
\large 对于提问赏析描写在刻画形象上的妙处的题目,要注意以下几个点:\par
\df{拆分提干}\nm 分析哪个词对应了什么形象; \\
\df{立体地塑造了...}.
\subsubsection{例题}
\large 
孙犁的文字,"寄至味于淡薄".请以水生夫妻炕头对话为例对此加以赏析.\\
\texttt{水生以事业以家国大事为重,而在外因为吃鱼这一日常小事想起家中的日常生活,想起妻子,符合战士身份(1);女人想起水生,因为她生活艰辛,牵挂家中的顶梁柱在外打仗生死未卜,因此她无论白天夜晚都会想念他,情深义重,符合乡村女子的个性身份和生活(1);而女人听闻丈夫要走,虽然不舍但仍然送他,体现她深明大义(从另一个角度分析),因此人物形象丰满立体(1);语言平淡但体现出夫妻之间情深义重以及以国家为重的深厚情感(1)。}
\nm \fangsong 
\\\centerline{原文见上}
\\女人歪在炕上,笑着问:``说真的,这八九年,你想起过我吗?'' ``想过。''``怎么想法?''她逼着问
\\``临过平汉路的那天夜里,我宿在一家小店,小店里有个鱼贩子是咱们乡亲。我买了一包小鱼下饭,吃着那鱼,就想起了你。''
\\ ``胡说。还有吗?''``没有了。你知道我是出门打仗去了,不是专门想你去了。''
\\ ``我们可常常想你,黑夜白日。''她突然支着身子坐起来,问:``你能在家住几天?''
\\ ``就这一晚上。我是请假绕道来看你的。''``为什么不早些说?''``还没顾着啊!''
\\ 女人呆了。她低下头去,无力地仄在炕上。过了半天,她说:``那么就赶快休息吧,明天我撑着冰床子去送你。''〔乙〕
\songti
\newpage

\section{说理与议论}

\subsection{『举例』``请分析举例论证的效果''}
在这类题目中, 答题时必须包括\par 
\df{举了什么例子}, \par 
\df{例子阐述了什么观点}以及\par
\df{这样写有什么表达效果}\nm 典型事例,具有典型性.
\subsubsection{例题}
\large \songti
分析第一段以颜回为例说理的作用(3分).\\
\texttt{本段例举了颜回虽屈居于陋巷, 无施于事, 无见于言, 却不妨碍众人的推尊与后人的视之为圣人(1), 为前文只要修好身就能成为圣人的观点做了突出强调(1), 运用典型事例, 具有典型性(1).}
\nm \fangsong \par 草木鸟兽之为物,众人之为人,其为生虽异,而为死则同,一归于腐坏澌尽泯灭而已。而众人之中,有圣贤者,固亦生且死于其间,而独异于草木鸟兽众人者,虽死而不朽,逾远而弥存也。其所以为圣贤者,修之于身,施之于事,见之于言,是三者所以能不朽而存也。修于身者,无所不获;施于事者,有得有不得焉;其见于言者,则又有能有不能也。施于事矣,不见于言可也。自诗书史记所传,其人岂必皆能言之士哉?修于身矣,而不施于事,不见于言,亦可也。孔子弟子,有能政事者矣,有能言语者矣。若颜回者,在陋巷曲肱饥卧而已,其群居则默然终日如愚人。然自当时群弟子皆推尊之,以为不敢望而及。而后世更百千岁,亦未有能及之者。其不朽而存者,固不待施于事,况于言乎?\par 予读班固艺文志,唐四库书目,见其所列,自三代秦汉以来,著书之士,多者至百余篇,少者犹三、四十篇,其人不可胜数;而散亡磨灭,百不一、二存焉。予窃悲其人,文章丽矣,言语工矣,无异草木荣华之飘风,鸟兽好音之过耳也。方其用心与力之劳,亦何异众人之汲汲营营? 而忽然以死者,虽有迟有速,而卒与三者同归于泯灭,夫言之不可恃也盖如此。今之学者,莫不慕古圣贤之不朽,而勤一世以尽心于文字间者,皆可悲也!\par 东阳徐生,少从予学,为文章,稍稍见称于人。既去,而与群士试于礼部,得高第,由是知名。其文辞日进,如水涌而山出。予欲摧其盛气而勉其思也,故于其归,告以是言。然予固亦喜为文辞者,亦因以自警焉。
\songti


\subsection{『类比』``类比说理的特色是?''}
对于考察类比说理的题目需注意\par
\df{类比的本体和喻体}\par
\df{表达效果}\nm (抽象概念具象化)(生动形象)\par
\df{与文章主旨的关联}


\subsection{『说服』``哪个论证更有说服力?"}
\large 对于提问说服力的题目,要注意以下三个点:\par
\df{角度全面}, \\
\df{语言严谨}, \\
\df{论证手法}.\par
如果原文有比较,答题过程中也应该注意比较.


\newpage

\section{结构,思想,评价与概括}

\subsection{『构思』``请从构思角度赏析''}
对于提问构思的题目尤其要注意情感线索.例如,在文章<良宵>中,"鹅"这一物象贯穿全文,是一个十分重要线索.对于"鹅"的分析要注意其出现的段落与情感线索.我们需要考虑它的来历,经历,与它的得失是如何推动情节发展的.:\par
\df{结构}\par
\df{主旨}\par
\df{为什么能突出主旨}\\
\subsubsection{例题}
作品围绕``窥看自己的心魂''与自我对话,请从构思角度对此作赏析.\\
\texttt{本文首先提出心中难解的关于生死和写作的三个问题,并对自我思考过程展开反思.从尝试写作的试探,到沉迷写作的焦虑,最后到在对写作的坚持中获得与自己和解与释然,有深度.}
\subsubsection{例题}
小说中的"鹅"在全文构思中有重要作用,请加以赏析.
\texttt{鹅本是老太太捡来的,陪伴了她十三年,是她的精神寄托.}


\nm \fangsong 
\\\centerline{我与地坛}\\\centerline{史铁生}
\\设若有一位园神,他一定早已注意到了,这么多年我在这园里坐着,有时候是轻松快乐的,有时候是沉郁苦网的,有时候优哉游哉,有时候栖惶落寞,有时候平静而且自信,有时候又软弱,又迷茫。其实总共只有三个问题交替着来骚扰我,来陪伴我。第一个是要不要去死?第二个是为什么活?第三个,我干嘛要写作? 现在让我看看,它们迄今都是怎样编织在一起的吧.\\你说,你看穿了死是一件无需乎着急去做的事,是一件无论怎样耽搁也不会错过的事,便决定活下去试试?是的,至少这是很关健的因素。为什么要活下去试试呢?好像仅仅是因为不甘心,机会难得,不试白不试,腿反正是完了,一切仿佛都要完了,但死神很守信用,试一试不会额外再有什么损失。说不定倒有额外的好处呢是不是?我说过,这一来我轻松多了,自由多了。为什么要写作呢? 作家是两个被人看重的字,这谁都知道。为了让那个躲在园子深处坐轮椅的人,有朝一日在别人眼里也稍微有点光彩,在众人眼里也能有个位置,哪怕那时再去死呢也就多少说得过去了,开始的时候就是这样想,这不用保密,这些现在不用保密了。 \\我带着本子和笔,到园中找一个最不为人打扰的角落,偷偷地写。那个爱唱歌的小伙子在不远的地方一直唱。要是有人走过来,我就把本子合上把笔叼在嘴里。我怕写不成反落得尴尬。我很要面子。可是你写成了,而且发表了。人家说我写的还不坏,他们甚至说:真没想到你写得这么好。我心说你们没想到的事还多着呢。我确实有整整一宿高兴得没合眼。我很想让那个唱歌的小伙子知道,因为他的歌也毕竟是唱得不错。我告诉我的长跑家朋友的时候,那个中年女工程师正优雅地在园中穿行;长跑家很激动,他说好吧,我玩命跑。你玩命写。\\这一来你中了魔了,整天都在想哪一件事可以写,哪一个人可以让你写成小说。是中了魔了,我走到哪儿想到哪儿,在人山人海里只寻找小说,要是有一种小说试剂就好了,见人就滴两滴看他是不是一篇小说,要是有一种小说显影液就好了,把它泼满全世界看看都是哪儿有小说,中了魔了,那时我完全是为了写作活着。结果你又发表了几篇,并且出了一点小名,可这时你越来越感到恐慌。我忽然觉得自己活得像个人质,刚刚有点像个人了却又过了头,像个人质,被一个什么阴谋抓了来当人质,不走哪天被处决,不定哪天就完蛋。你担心要不了多久你就会文思枯竭,那样你就又完了。凭什么我总能写出小说来呢?凭什么那些适合作小说的生活素材就总能送到一个截瘫者跟前来呢?人家满世界跑都有枯竭的危险,而我坐在这园子里凭什么可以一篇接一篇地写呢?\\你又想到死了。我想见好就收吧。当一名人质实在是太累了太紧张了,太朝不保夕了。我为写作而活下来,要是写作到底不是我应该干的事,我想我再活下去是不是太骨傻气了?你这么想着你却还在绞尽脑汁地想写。我好歹又拧出点水来,从一条快要晒干的毛巾上。恐慌日甚一日,随时可能完蛋的感觉比完蛋本身可怕多了,所谓不怕贼偷就怕贼惦记,我想人不如死了好,不如不出生的好,不如压根儿没有这个世界的好。可你并没有去死。我又想到那是一件不必着急的事。可是不必着急的事并不证明是一件必要拖延的事呀?你总是决定活下来,这说明什么?是的,我还是想活。\\人为什么活着?因为人想活着,说到底是这么回事,人真正的名字叫作:欲望。可我不怕死,有时候我真的不怕死。有时候, ---说对了。不怕死和想去死是两回事,有时候不怕死的人是有的,一生下来就不怕死的人是没有的。我有时候倒是伯活。可是怕活不等于不想活呀?可我为什么还想活呢?因为你还想得到点什么、你觉得你还是可以得到点什么的,比如说爱情,比如说,价值之类,人真正的名字叫欲望。这不 对吗? 我不该得到点什么吗?没说不该。可我为什么活得恐慌,就像个人质?后来你明自了,你明白你错了,活着不是为了写作,而写作是为了活着。你明自了这一点是在一个挺滑稽的时刻。那天你又说你不如死了好,你的一个朋友劝你:你不能死,你还得写呢,还有好多好作品等着你去写呢。这时候你忽然明白了,你说:只是因为我活着,我才不得不写作。或者说只是因为你还想活下去,你才不得不写 作。是的,这样说过之后我竟然不那么恐慌了。就像你看穿了死之后所得的那份轻松?一个人质报复一场阴谋的最有效的办法是把自己杀死。我看出我得先把我杀死在市场上,那样我就不用参加抢购题材的风潮了。你还写吗?还写。你真的不得不写吗?人都忍不住要为生存找一些牢靠的理由。你不担心你会枯竭了?我不知道,不过我想,活着的问题在死前是完不了的。\\这下好了,您不再恐谎了不再是个人质了,您自由了。算了吧你,我怎么可能自由呢?别忘了人真正的名字是:欲望。所以您得知道,消灭恐慌的最有效的办法就是消灭欲望。可是我还知道,消灭人性的最有效的办法也是消灭欲望。那么,是消灭欲望同时也消灭恐慌 呢?还是保留欲望同时也保留人生?我在这园子里坐着,我听见园神告诉我,每一个有激情的演员都难免是一个人质。每一个懂得欣赏的观众都巧妙地粉碎了一场阴谋。每一个乏味的演员都是因为他老以为这戏剧与自己无关。 每一个倒霉的观众都是因为他总是坐得离舞合太近了。\\我在这园子里坐着,园神成年累月地对我说:孩子,这不是别的,这是你的罪孽和福扯。
\songti

\subsection{『判词』人物传记特色评价词}
我们经常能够在做人物传记题目时遇到诸如``坚正''之类的评价性词语, 进而要求分析人物. 这里要注意小词放大的技巧, 例如: \LARGE 坚 \large 持操守, \LARGE 正\large 直讲义. 注意要点\par
\df{小词放大}\par
\df{事例一概括与品格一}\par
\df{事例概括二与品格二}

\subsection{『事迹』``请概括人物事迹''}
\large
概括人物事迹时需要答道\par
\df{人物事迹一}\par
\df{人物事迹二}\par
\df{人物品格}

\subsection{『思情』``请分析全文的思想情感''}
分析时要特别注意题干中是说\df{具体分析}还是\df{大致概括}.



\subsection{『情变』``作者情感态度是怎么转变的?''}
对于情感题, 我们应注意\par
\df{全文逐段体会}\par
\df{找议论的句子}\nm (判断, 表推测)\par
\df{直接找情感关键词}

\subsubsection{例题}
\\小说第1段她仅仅是对他``点头致意'',第15段她却``直直地冲他微笑''.有人认为这一转变缺少铺垫,不切实际.对此你是否认同,说说你的看法. 
\\ \texttt{我不认同.奥一开始只是点头致意,表达了最基本的礼仪;随后因被窥视而感到不满;后来又因为里桑谨慎礼貌的行为留下了美好的印象,由此推测他是一位会忍耐,尊重的男人;最后她又因为他的骑士风度,从而想象不同情景下的他,产生海市蜃楼般的美好幻想.由此可见最后"直直地冲他微笑"有着充分的铺垫,是自然的心理变化的体现.(全文逐段体会)}
\subsubsection{例题}
\\第9段画线句运用动作细节和语言细节,刻画了老太太怎样的心理变化过程。请简要分析。
\\ \texttt{老太太首先"拾"炊具,"捋"衣袖,"抽"起来,后"心理麻幽幽的",并"撒","默默"让他走,最后"不再搭理",体现了老太太由爱鹅被杀的愤怒到心疼男孩的不忍,再到心中矛盾纠结,最后无可奈何.}
\nm \fangsong 
\\\centerline{玻璃边界}
\\ \begin{center} 1 \end{center} \\当利桑德罗与奥德丽目光相遇时,她点头致意,就像出于礼貌问候一位餐厅服务生,比问候公寓楼的门卫还要少一分热情……利桑德罗已经擦净了第一扇玻璃窗,正是奥德丽办公室的那扇,随着他慢慢除去灰尘形成的薄膜,她逐渐显现,起初遥远而朦胧,随后便一点点靠近,由于玻璃越来越清澈,她分毫未动却越来越近。就像调整相机的焦距,就像慢慢将她据为己有。\\玻璃的透明渐渐揭开她的面纱。办公室的灯光从身后照亮她的头部,为她栗色的头发笼罩上一层麦田般的柔美和动感,麦穗与如饰带般落于颈后的美丽金黄的麻花辫纠缠。光线聚集于后颈,当她将浅色的柔软辫子拨到一边时,后颈上的光照亮了从背部蜿蜒向上的每一根金黄的绒毛,就像一把种子,即将在编织的发束里找到土壤。 \\她伏案工作着,对他无动于衷,对他人的工作无动于衷,那种卑躬屈膝的手工劳动,与她的截然不同。她正努力为百事可乐找一句精彩的、引人注目、朗朗上口的广告语。他感到不自在,担心自己手臂在玻璃上的挥舞使她分神。如果她抬起头,会是因为工人的打扰而一脸愤怒吗? \\如果她再次看他,会用什么样的眼神? \\``上帝啊,''她低声自言自语,``他们提醒过我会有工人来。但愿这个男人没有在观察我。我感觉在被窥视。我有点生气了,没法集中精力。'' \\ \begin{center} 2 \end{center} \\她抬起头,碰上了利桑德罗的目光。她想要发怒却没能做到。那张脸上有种东西令她吃了一惊。一开始,她没有注意他外表的细节。令她战栗的是别的东西。某种她几乎从未在男人身上见过的东西。她在自己的词汇表里拼命搜寻,作为一个以遣词造句为职业的人,她寻找着一个词汇,来形容这个办公室玻璃清洁工的态度和面孔。\\在一闪念间她找到了——礼貌。在这个男人的身上,在他的态度、距离感、点头的方式与奇妙地混杂着忧伤和欢乐的目光中的那种东西,是礼貌,难以置信地毫无粗俗的痕迹。 \\“这个男人,”她想,“他绝不会在凌晨两点钟歇斯底里地打电话请求原谅,他会忍耐。他会尊重我的孤独,我也会尊重他的。” \ “这个男人会为你做什么?”她马上自问。“他会请我吃晚饭,然后送我到家门口。他不会让我在夜里独自叫出租车离开。” \\正当她抬起目光、神不守舍之时,他在转瞬之间看见了她深邃的栗色大眼睛。他马上垂下目光,继续工作,但与此同时他想起她微笑了。这是他的想象吗?还是真的?他鼓起勇气望向她。女人对他微笑,非常短促,非常礼貌,然后就低下头继续工作。\\一个眼神足矣。他没想到会在一个美国女人的眼睛里看到忧郁。人们说她们都很坚强,很自信,很专业,很守时,不是说所有的墨西哥女人都软弱、摇摆、随性、拖沓,不,完全不是。问题在于,一个会在星期六来工作的女人可能是各种样子,也许温柔,也许亲热,但唯独不该是忧郁的。利桑德罗清楚地在这个女人的眼神里看到了忧郁。她怀着悲伤,也怀着渴望。她渴望着。这是她的眼神所诉说的:“我想要某种缺失的东西。” \\奥德丽不必要地把头压得很低,好躲进纸张文件中。这太荒唐了。她难道要爱上大街上第一个擦肩而过的男人,只为了和丈夫彻底分手,让他吸取教训,只是因为纯粹的反弹效应?那个工人很英俊,这是糟糕之处,他有着不寻常的几乎令人感到冒犯的骑士风度,不合时宜,仿佛在滥用他的弱势地位,但他同时有着明亮的眼睛,眼里流露出的悲伤和喜悦同样浓烈,他的皮肤呈橄榄色,鼻子短而尖,鼻翼翕动着,身形修长,卷发,年轻,胡须厚重。与他的丈夫迥然不同。他是——她又一次露出微笑——一个海市蜃楼。 \\ \begin{center} 3 \end{center} \\他也对她回以微笑。他的牙齿坚硬、洁白。利桑德罗想到,他极力避开了会使他在当他还是个有志青年时认识的人面前降低身份的工作。他曾接下一份在弗克拉尔餐馆做服务员的差事,当他不得不为一桌中学老同学服务时,场面十分难堪。所有人都事业有成,除了他。他令他们难堪,他们也令他难堪。他们不知道该怎么称呼他,对他说些什么。还记得和西蒙·玻利瓦尔队比赛的时候你进的那个球吗?这是他听到的最友善的话了,随之而来的是一阵令人尴尬的沉默。 \\他做不了办公室文员,从中学三年级起他就辍学了,不会速记法也不会用打字机。做出租车司机更不行。他嫉妒比他有钱的乘客,看不起比他穷的,墨西哥城混乱的交通令他发狂,让他火冒三丈,暴跳如雷,不停骂娘,变成各种自己不喜欢的样子……超市售货员,加油站雇员,他什么都做过,那是自然。不幸的是现在连这样的差事都没有了。他感恩能获得这份来美国的工作,感恩此刻正直视着他的这个女人的眼睛。 \\他并不知道,她不仅在看着他,也在想象他。她先他一步。她想象着各种情境下的他。她把铅笔放到牙齿间。他会喜欢什么体育运动?他看起来很强壮,很健美。电影,演员,他喜欢电影、歌剧、电视剧吗?他是那种会透露电影结局的人吗?当然不是。这一眼就看得出来。她直直地冲他微笑。她会忍不住给伴侣讲出电影、侦探小说的结尾,除了自己的故事,因为永远不知道会怎么结束。\\她头脑中的想法他也许已经猜到一二。他多想能坦率地告诉她,我不一样,不要相信外表,我不应该在做这些,这不是我,我不是你想象的那样。可他不能对玻璃说话,他只能爱上玻璃上的光,而光可以穿过玻璃,触碰她,光是他们共同所有。 \\(有删节)\\【注释】①选自短篇小说集《玻璃边界》。小说集通过九个短篇故事,生动形象地刻画了墨西与美国这对邻居在长达两百年的历史演变中形成的恩怨。作品出版于《北美自由贸易协定》正式生效后不久,此时美墨两国间的贸易壁垒有所弱化,然而两国人民间的交流依然存在一些难以解决的问题。②卡洛斯·富恩特斯(1928年11月11日-2012年5月15日):墨西哥作家,他的作品深刻刻画了墨西哥的历史和现实。由于对欧美文明的了解和对拉美落后现状的认识,比起其他的的拉美作家,富恩特斯作品中存在着更强烈的忧患意识。
\blkx \begin{center} 良宵(节选) \end{center}\\顺势拎了把刷锅的炊具,捋起他衣袖就抽打起来。抽着抽着便瞧得他胳膊上全是银元大小的红斑,一圈连一圈,看得心里麻麻幽幽,索性撒了他,一屁股坐在灶台上,默默盯了他半晌,摆摆手说:“你走吧,走吧。以后不要再来了。”孩子一愣,没有动,只嘟囔道:“我奶奶死了……我杀了它祭祀……”老太太不再搭理他,转身回了屋,和衣躺下。

\songti \nm 

\subsection{『意蕴』``请品读加点词并体会其中意蕴''}
对于这类题目,需要注意\\
\df{词语的本意}\\
\df{语境}\\
\df{这段话上下文的语境}
\subsubsection{例题}
\fangsong
然而阿Q虽然常优胜,却直待\CJKunderdot{蒙}赵太爷打他嘴巴之后,这才出了名。\\
\texttt{"蒙"意为承蒙. 挨打也像荣幸蒙恩,形象地刻画出阿Q与看客以丧失人格为代价而换来趋炎附势的变态心理,体现了长期以来的奴性人格.}
\nm \songti




\end{document}